\chapter{Synthesis}

\vspace{-1cm}
\begin{center}
Giuseppe Littera, Eduard Hirsch and Paolo Dini
\end{center}

The component testing of the platform was performed only at an essential level, as reported in deliverable D3.2 \cite{INTERLACE_D32}, due to lack of time and funding. Although the CoreASIM model and the blockchain implementation both performed as required according to the basic tests, we did not have time or resources to develop a production-level framework to test the components of the system formally and in an automated way against the requirements.

Over the next year we will investigate how Cucumber,\footnote{\url{https://docs.cucumber.io/}} together with other testing methods, can be utilised to achieve a test plan which can be executed transparently and repeatedly. Cucumber is a tool which supports Behaviour-Driven Development (BDD \cite{wynne2017cucumber}), which can be seen as an advancement relative to plain Test-Driven Development (TDD \cite{beck2003test}) that adds ``Deliberate Discovery'' to the process. Deliberate Discovery can be summarized by the aim to ``deliberately seek and discover what development teams are ignorant about before implementations starts''. This goal/intention is very similar to the AS(I)M approach taken by INTERLACE, in other words the aim to get the implementation details right and clear for everyone.

In spite of some similarities, AS(I)Ms and Cucumber achieve this goal in rather different ways. Whereas AS(I)Ms rely on a very rigorous mathematical approach, Cucumber uses a natural language approach called Gherkin\footnote{\url{https://docs.cucumber.io/gherkin/}} connected with actual test implementations (in various languages\footnote{Java Example: \url{https://docs.cucumber.io/guides/10-minute-tutorial/\#see-scenario-reported-as-undefined}}). The aim of the work will be to combine such methods with the AS(I)M approach, thereby giving the strict mathematical definitions a connection to broadly accessible and understandable terms which can easily be verified by both (non-)technical readers as well as by automated tests. This more immediate development of component testing will be performed with own funds as a collaborative effort between SARDEX, SUAS, UH, and the open source community at large. We expect the component testing to be completed sometime in 2019. All updates will be shared on GitHub\footnote{\url{https://github.com/InterlaceProject/InterlaceBlockchain}} and/or on the project's website.\footnote{\url{https://www.interlaceproject.eu/}}

An interesting possible future PhD-level research project could be to investigate the feasibility of developing an ASIM-to-Gherkin compiler in order to run acceptance tests with Cucumber directly from the ASIM model implementation. If feasible, it could ultimately be integrated in the CoreASIM framework. The motivation is that although ASIMs are mathematically rigorous they are completely flexible in how the universe of terms of a given model is defined. In fact, the standard ASM practice is to use natural language-like expressions to model any system \cite{BoergerStaerk2003,BoergerRaschke2018}. Since also Gherkin is flexible in its use of natural language expressions, it should be in principle possible to tailor ASIMs to the Gherkin/Cucumber test logic and natural language properties. A compiler that maps the specification logic to Gherkin would then achieve the best of both worlds: both mathematical rigour and human readibility at the specification, modelling, and testing phases.

\newpage
