\chapter{Introduction}
\label{ch:Introduction}

\vspace{-1cm}
\begin{center}
Giuseppe Littera, Eduard Hirsch and Paolo Dini
\end{center}

The component testing of the platform was performed only at an essential level, as reported in deliverable D3.2 \cite{INTERLACE_D32}, due to lack of time and funding. Although the basic testing of the CoreASIM model and the blockchain implementation both performed as required, we did not have time or resources to develop a production-level framework to test the components of the system formally and in an automated way against the requirements.

\textcolor{blue}{This report will explain how Cucumber\footnote{\url{https://docs.cucumber.io/}}, together with other testing methods will be facilitated in order to achieve a test-plan which can be executed transparently and repeatedly. Cucumber is a tool which supports Behaviour-Driven Development (BDD \cite{wynne2017cucumber}) which can be seen as an advancement to plain Test Driven Development (TDD \cite{beck2003test}) adding "Deliberate Discovery" to the process.\\\\
"Deliberate Discovery" \cite{wynne2017cucumber} can be summarized by the aim to "deliberately seek and discover what development teams are ignorant about before implementations starts", thus, a goal/intention which is very similar to the AS(I)M approach taken by INTERLACE. In other words to get the implementation details right and clear to everyone.\\\\
Nevertheless, AS(I)Ms and Cucumber are achieving this goal in quite different way. Whereas AS(I)Ms are taking a very rigorous mathematical approach, Cucumber is going a different way using natural language approach called Gherkin\footnote{\url{https://docs.cucumber.io/gherkin/}} connected with actual test-implementations (in various languages\footnote{Java Example: \url{https://docs.cucumber.io/guides/10-minute-tutorial/\#see-scenario-reported-as-undefined}}). The aim for this report is to combine such methods with the ASM approach giving the strict mathematical definitions a connection to broadly accessible and understandable terms which can easily be verified by both (non-)technical readers as well as by automated tests.
}

\textcolor{red}{One possible way to do that will be to develop an ASIM-to-Gherkin compiler in order to run acceptance tests with Cucumber.}  This work will be performed with own funds as a collaborative effort between SARDEX, SUAS, and the open source community at large. We expect the component testing to be completed sometime in 2019. All updates will be shared on GitHub\footnote{\url{https://github.com/InterlaceProject/InterlaceBlockchain}} and/or on the project's website.\footnote{\url{https://www.interlaceproject.eu/}} 

\newpage
